\section{Introducción}\label{intro}
\begin{frame}[fragile]{Introducción}
  \textit{\LARGE Bienvenidos!!!}


  En esta presentación daremos a conocer el análisis que se realizó para la confección de nuestro Proyecto de Ciencia de Datos,
  el cual tiene como objetivo exponer el estudio e investigación de cuatro productos:
\begin{itemize}
  \item Cerveza
  \item Cebolla
  \item Refresco Instantáneo
  \item Refresco Gaseado
\end{itemize}
  
\end{frame}

\subsection{Importación de datos y Bibliotecas Utilizadas}

\begin{frame}{Importación de los Datos}
Para la confección de nuestro proyecto utilizamos un archivo \textbf{(.json)} donde ubicamos por cada línea 
de código categorías de datos de cada uno de los productos. Este archivo \textbf{(.json)} conforma toda nuestra Base de Datos
utilizada.
\end{frame}


\subsection{Importación de datos y Bibliotecas Utilizadas}
\begin{frame}{Bibliotecas Utilizadas}
La utilidad de varias Bibliotecas en el ambiente de \textbf{Python} fue escencial
para el desarrollo y la visualización de los datos. Importamos varias Bibliotecas dependiendo de su uso 
en nuestro proyecto.
Estas fueron:
\begin{itemize}
  \item Pandas
  \item Matplotlib
  \item Numpy
  \item Folium
  \item Plotly
  \item Seaborn 
\end{itemize} 
\end{frame}


