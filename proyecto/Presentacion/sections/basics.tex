\section{Recolección y Recopilación de Datos}
\begin{frame}[fragile]{Confección}
    En nuestro proyecto, recopilamos datos con respecto a la ubicación e imágenes de cada uno de los lugares que visitamos en el municipio de Marianao, el municipio experimento de nuestro proyecto. 
Las imágenes se encuentran como prueba de la recopilación de los datos y se habilitó una carpeta donde se pueden ver cada una de ellas, en caso de ser necesario.  
La geolocalización es característica en nuestros datos, ya que se puede presenciar el lugar exacto de donde se extrajo el dato, lo que ayuda a que nuestra información llegue con más peso y valor al espectador. 
Entre otras categorías, se encuentran ya rasgos y datos más específicos de cada producto que están presentes en las próximas tablas.
\end{frame}


\newpage
\begin{frame}[fragile]{Tablas}
    \begin{center}
        \textbf{Cervezas}
        \begin{tabular}{|c|c|c|c|c|c|c|}
            \hline
             Marca & Precio & Coloración & Localización & Municipio\\
            \hline
             Colonia & 160 & Clara & 23.08565, -82.42872 & Marianao\\
            \hline
        \end{tabular}

        \begin{tabular}{|c|c|}
            \hline
             Foto & País\\
            \hline
             /fotos/foto1.jpg & Brasil\\
            \hline
        \end{tabular}
  


        \textbf{Refresco Gaseado}
        \begin{tabular}{|c|c|c|c|c|c|c|}
            \hline
             Marca & Sabor & Envase & Precio & Localización & Municipio\\
            \hline
             Pepsi & Cola & Lata & 150 & 23.0720,-82.4324 & Marianao\\
            \hline
        \end{tabular}

        \begin{tabular}{|c|c|}
            \hline
             Imagen & País\\
            \hline
             /fotos/foto52.jpg& EU\\
            \hline
        \end{tabular}
    \end{center}
\end{frame}


